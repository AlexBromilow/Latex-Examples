\documentclass{article}
\usepackage[utf8]{inputenc}
\usepackage[backend=biber,style=numeric]{biblatex}
\usepackage{cleveref}
\usepackage{Stylesheets/MyStylesheet}
\includeonly{Subsections/Subsection2}

\newtheorem{theorem}{Theorem}

\begin{document}

\section{New Commands}
$\displaystyle\frac{\partial z}{\partial x}$ vs $\pd{z}{x}$

\section{Conditional Formatting}
This is question 1. \solution{This is my solution}
\section{Stylesheets}
Look at markup for how Stylesheets are used
\section{Multiple Files}
\subsection{This is subsection 1}

Test 1 %Use input to not start on new page
\subsection{This is subsection 2}

Test 2

\section{Be Semantic}

Compare \textit{this} and \emph{this}.

\begin{theorem}
    \label{thm1}
    Compare \textit{this} and \emph{this}.
\end{theorem}

\large{\bf This is some big text}.

\subsection{This is some big text}

\section{Cleveref}
\begin{equation}
    \label{pd1}
    \pd{z}{x}
\end{equation}
\begin{equation}
    \label{pd2}
    \pd{z}{x}
\end{equation}
\begin{equation}
    \label{pd3}
    \pd{z}{x}
\end{equation}
\begin{equation}
    \label{pd4}
    \pd{z}{x}
\end{equation}

I can now refer to \cref{pd1}. I can also do \cref{pd1,pd2,thm1,pd3}

\section{Template Files}

\end{document}

