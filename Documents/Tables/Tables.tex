\documentclass{article}
\usepackage[utf8]{inputenc}
\usepackage[backend=biber,style=numeric]{biblatex}
\usepackage{array, multirow}
\usepackage[table]{xcolor}

\setlength{\tabcolsep}{18pt} %Gap before text starts
\renewcommand{\arraystretch}{1.5} %Cell height scaling
\setlength{\arrayrulewidth}{0.5mm}%Table border thickness
\arrayrulecolor{blue} %Table border color
\newcolumntype{s}{>{\columncolor{blue!20}} c}

\begin{document}

\section{Tables}
\begin{tabular}{|l|b{4cm}|}
    \hline
    This & is                                                       \\
    \hline
    a    & table and it has a really really really really long cell \\
    \hline
\end{tabular}

\vspace{1in}

\begin{tabular}{|c|s|c|}
    \hline
    \multicolumn{3}{|c|}{A multiple column}                       \\
    \hline
    \rowcolor{yellow!50} 1            & 2 & \cellcolor{black!10}3 \\
    \hline
    \multirow{2}{2cm}{A multiple row} & 5 & 6                     \\
                                      & 8 & 9                     \\
    \hline
\end{tabular}

\vspace{1in}

\rowcolors{2}{red!30}{green!30}
\begin{table}[!h]
    \label{Example}
    \caption{Table examples}
    \begin{tabular}{|c|c|c|}
        \hline
        1  & 2  & 3  \\
        \hline
        4  & 5  & 6  \\
        \hline
        7  & 8  & 9  \\
        \hline
        10 & 11 & 12 \\
        \hline
        13 & 14 & 15 \\
        \hline
    \end{tabular}
\end{table}

Check out table \ref{Example}

\end{document}